%%%%%%%%%%%%%%%%%%%%%%%%%%%%%%%%%%%%%%%%%%%%%%%%%%%%%%%
% A template for Wiley article submissions developed by 
% Overleaf for the Overleaf-Wiley pilot which ran 
% during 2017 and 2018.
% 
% This template is no longer supported, but is provided
% for historical reference. Last updated January 2019.
%
% Please note that whilst this template provides a 
% preview of the typeset manuscript for submission, it 
% will not necessarily be the final publication layout.
%
% Document class options:
% =======================
% blind: Anonymise all author, affiliation, correspondence
%        and funding information.
%
% lineno: Adds line numbers.
%
% serif: Sets the body font to be serif. 
%
% twocolumn: Sets the body text in two-column layout. 
% 
% num-refs: Uses numerical citation and references style 
%           (Vancouver-authoryear).
%
% alpha-refs: Uses author-year citation and references style
%             (rss).
%
% Using other bibliography styles:
% =======================
%
% To specify a different bibiography style
%
% 1) Do not use either num-refs or alpha-refs in documentclass.
% 2) Load natbib package with the options set as needed.
% 3) Use the \bibliographystyle command to specify the style
% 
% Included NJD styles are: 
%   WileyNJD-ACS
%   WileyNJD-AMA
%   WileyNJD-AMS
%   WileyNJD-APA
%   WileyNJD-Harvard
%   WileyNJD-VANCOUVER
%
% or you may upload an alternative .bst file 
% (if requested by the journal).
%
% Examples:
% =======================
%% Example: Using numerical, sort-by-authors citations.
\documentclass[num-refs]{wiley-article}

%% Example: Using author-year citations and anonymising submission
% \documentclass[blind,alpha-refs]{wiley-article}

%% Example: Using unsrtnat for numerical, in-sequence citations
%\documentclass{wiley-article}
%\usepackage[numbers]{natbib}
%\bibliographystyle{unsrtnat}

%Citation style from another project
%\usepackage[citestyle=numeric,style=numeric,backend=biber]{biblatex}
%\addbibresource{rifporphyrin.bib}
%% Example: Using WileyNJD-AMA reference style and superscript
%%          citations, two-column and serif fonts for AIChE
\documentclass[serif,twocolumn,lineno]{wiley-article}
\usepackage[super]{natbib}
\bibliographystyle{WileyNJD-AMA}
% \makeatletter
% \renewcommand{\@biblabel}[1]{#1.}
% \makeatother

% Add additional packages here if required
\usepackage{siunitx}

% Update article type if known
\papertype{Original Article}
% Include section in journal if known, otherwise delete
\paperfield{Organic chemistry synthesis}

\title{Synthesis of \texorpdfstring{H\textsubscript{2}}-TPP and its metallation to Zn-TPP}

% List abbreviations here, if any. Please note that it is preferred that abbreviations be defined at the first instance they appear in the text, rather than creating an abbreviations list.
\abbrevs{\texorpdfstring{H\textsubscript{2}}-TPP, 5,10,15,20-mesotetraphenylporhyrin; DEF, doesn't ever fret; GHI, goes home immediately.}

% Include full author names and degrees, when required by the journal.
% Use the \authfn to add symbols for additional footnotes and present addresses, if any. Usually start with 1 for notes about author contributions; then continuing with 2 etc if any author has a different present address.
\author[1\authfn{1}]{Matteo Finco}


% Include full affiliation details for all authors
\affil[1]{DiSC, Università degli Studi di Padova, Padova, Italy, 35131, Italy}

\corremail{matteo.finco.3@studenti.unipd.it}
% Include the name of the author that should appear in the running header
\runningauthor{Matteo Finco}

\begin{document}

\begin{frontmatter}
\maketitle

\begin{abstract}

% Please include a maximum of seven keywords
\keywords{tetraphenylporphyrin, TPP, Zn-TPP}
\end{abstract}
\end{frontmatter}

\section{Introduction}
Porhyrins represent an obiquitous class of molecules that plays a crucial role biochemistry for oxygen transport\cite{hardison_evolution_2012}, storage\citep{kendrew_three-dimensional_1958} and catalysis as well as electron transfer\citep{keilin_cytochrome_1925}.\\
Therefore, extensive efforts have been put into the exploration of the functionalities obtainable by artificially synthesized porphyrins.\\
The biomimetic electrocatalysis\cite{facchin_oxygen_2021}\cite{liang_porphyrin-based_2021} and cancer treatment and diagnosis\cite{wang_recent_2021} represent two outstanding examples among all the use cases in which porphyrins have been successfully deployed.
Numerous works elaborate the further functionalization of \textit{meso}-tetraphenylporphyrin\cite{silva_porphyrins_2006} in order to realize functional materials.
However, to date the scale up in complex porphyrins production and broad commercial application is prevented by the production costs which are primarily led up by the expensiveness of reactants, inefficient and unreliable synthesis.
In the following work, a simple synthesis route for 5,10,15,20-mesotetraphenylporhyrin is detailed and its products are critically discussed.
\twocolumn
\section{Methods}
\section{Results and Discussions}
\section{Conclusions}
\section{Experimental}

\section*{acknowledgements}
The contributions of former students in collecting the experimental characterization of their sample has been crucial for validating the synthesis proposed in this work.
\section*{conflict of interest}
The author reports to have a personal interest in the positive outcome of the research work.
\section*{Supporting Information}
A GitHub repository hosts the complete file collection produced as part of this coursework.
It can be found at https://github.com/EC-Finco/Porphyrins-report.
\printendnotes

% Submissions are not required to reflect the precise reference formatting of the journal (use of italics, bold etc.), however it is important that all key elements of each reference are included.
\bibliography{rifporphyrin}

\graphicalabstract{example-image-1x1}{Please check the journal's author guildines for whether a graphical abstract, key points, new findings, or other items are required for display in the Table of Contents.}

\end{document}